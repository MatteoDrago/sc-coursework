\section{MATLAB Approach}

We decided to implement \textit{collaborative filtering} with the \textit{user-based} approach: under the assumption that people who exhibits same preferences in the past will likely do the same in future, we want to find a way to group these users, or to find a measure of similarity between them. One way to do it (which is also the one we adopted) is make use of the \textbf{jointly rated items}. 

As a consequence, in order to get the similarity between two users we need first to collect all the items that both have reviewed. Then, using $u_1$ and $u_2$ to refer to the different users, and $\mathcal{I}$ for the set of jointly rated items, we can find the similarity using:
\begin{equation}
	sim(u_1,u_2) = \frac{\sum_{i \in \mathcal{I}} (r_{u_1,i}-\bar{r}_{u_1}) (r_{u_2,i}-\bar{r}_{u_2})} {\sqrt{(r_{u_1,i}-\bar{r}_{u_1})^2}\sqrt{(r_{u_2,i}-\bar{r}_{u_2})^2}}
	\label{eq_1}
\end{equation}
where $\bar{r}_{u_j}$ is the average rate of the j-th user and $r_{u_j,i}$ is the rate given by the j-th user to the i-th item.

The measure just evaluated takes the name of Pearson Coefficient, which is demonstrated to obtain significant performances when it comes to collaborative filtering implementations; it takes value from -1 which correspond in complete uncorrelation (what we simply call in jargon "difference of tastes"), to +1 which on the contrary means that users have tastes exactly equal.
 
It should be clear that in this way we'll eventually obtain a matrix with only ones in the diagonal (every user is completely similar to himself). However, we cannot fill all the cells of the matrix: of course, when the set of jointly rated items is empty, similarity won't be computed. We'll delve more into the details of this later in this report, when we'll talk about the \textbf{cold start problem}.     

Our first idea was to evaluate this matrix with MATLAB, given that it is naturally optimized for operations with matrices and vectors. Previous works with this programming language suggest us that, despite the dimension of the database, we could obtain significant and fast results.

For this problem, the training set contained 16045651 entries and was made of three columns: \textbf{userID}, \textbf{itemID}, \textbf{rating}; we imported everything into a \code{table} data structure (note that the CSV was ordered by userID). Then, in order to study if the use of MATLAB was effective, we started with the evaluation of the \textbf{similarity matrix} for a small subset of user (almost 9000 out of more than 135000). Even we this relatively small subset we had to spend a considerable amount of time searching for time bottlenecks with the \textit{code-profiler} provided by the MATLAB environment. It is known that most of time problems with this programming language comes up when implementing nested for-loop structures. So, we tried to minimize the time taken by the script using array and matrix operations instead of loops, when possible.

Different tests on time performances showed real improvements when consecutive portions of the dataset were stored on a matrix: in this way we speeded up the retrieval of the list of items to pass to the method \code{evaluatePearson}. In the following, the pseudo-code of the method:
\begin{algorithm}
	\caption{evaluatePearson($user_1$,$user_2$)}
	\begin{algorithmic} 
		\STATE $avg_1 \leftarrow avgItemsRateUser1$
		\STATE $avg_2 \leftarrow avgItemsRateUser2$
	
		\IF{($max(user_1.item) < min(user_2.item) \; || \; max(user_2.item) < min(user_1.item))$}
		\RETURN NaN
		\ENDIF
		
		\STATE $c \leftarrow 1$
		
		\IF{($size(user_2)$ < $size(user_1)$)}
		\FOR{i in [1,$size(user_2)$]}
		\IF{($user_2.item(i)$ in $user_1.item$)}
		\STATE{$r_1(c) \leftarrow rateUser1ItemI$}
		\STATE{$r_2(c) \leftarrow rateUser2ItemI$}
		\STATE{$c \leftarrow c + 1$}
		\ENDIF
		\ENDFOR
		\ELSE
		\FOR{i in [1,$size(user_1)$]}
		\IF{($user_1.item(i)$ in $user_2.item$)}
		\STATE{$r_1(c) \leftarrow rateUser1ItemI$}
		\STATE{$r_2(c) \leftarrow rateUser2ItemI$} 
		\STATE $c \leftarrow c + 1$
		\ENDIF
		\ENDFOR
		\ENDIF
		
		\RETURN{\textit{pearson}($r_1$,$\bar{r}_1$,$r_2$,$\bar{r}_2$)}
	\end{algorithmic}
\end{algorithm} 
The most important optimization step in this case is in the two if-else statements. If we consider that each item as a numeric ID, this means that if the maximum itemID rated by $user_1$ is less than the minimum rated by $user_2$ (or viceversa), there are no common items and we return NaN. Then, in order to minimize the number of comparisons we iterate on the user that reviewed less items, and we temporary save all the ratings for the items in common in two different arrays. Finally, we return the pearson coefficient obtained with Eq. \ref{eq_1}. Other details on array manipulation can be found in the \code{matlab-test} folder.

After all this time-performances optimization, we encountered another problem: if we build the matrix in this way (the only one possible if we want to retrieve similarity measure in a faster way) we will end with a matrix of dimensions 135000x135000. Unfortunately, this would take with MATLAB 135.8GB of disk space, which exceeds the array size that it could manage (quoting the MATLAB compiler \textit{"Creation of
	arrays greater than this limit may take a long time and cause MATLAB to become unresponsive."}).

We concluded that MATLAB wasn't suited for this type of operations and we decided to change architecture: nevertheless, given that this algorithm worked for small sets of users, we decided to use it cross-checking the coefficient values obtained with the following methods.

\clearpage