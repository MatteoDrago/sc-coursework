\section{Theoretical Introduction}

A recommender system is a technology that involves two side of the same coin: \textit{users} and \textit{items}. Items could be products, movies, events, articles that are going to be recommended to users such as: customers, app users, visitors. We can relate the concept of recommender system to a real life situation; imagine a small shop or boutique, then a merchant knows personal preferences of everyday customers. Its high quality recommendations make customers satisfied and increase profits and visibility of the shop. When dealing with online market places (Netflix, Amazon, Ebay or Asos), personal recommendations, suited up for each particular and different user, can be generated by an "artificial merchant": the \textbf{recommender system}.

There are several definitions of recommender system in the literature and in the \textit{Internet}. For example Wikipedia fives the following definition of recommender system: 

\textit{"A recommender system or a recommendation system (sometimes replacing "system" with a synonym such as platform or engine) is a subclass of information filtering system that seeks to predict the "rating" or "preference" that a user would give to an item".}

Such definition, on our opinion, is not completely correct, or better, a recommender system is more than what stated by this definition. A recommender system is an assistant because it can help you in order to make the right decision when you're going to buy something. Moreover, it knows your preferences and tastes; therefore it's not only a machine or an IT system. A more proper definition could be: 

\textit{"A recommender systems is a system that help users discover items they may like, based on a sort of personal knowledge of each user".}

Formally, a recommender system a type of information filtering system which filters items based on a personal profile. It differs from many other information filtering systems, such as standard search engines and reputation systems, because its recommendation are personally conducted based on the specific user or group of users. The main goal for a recommender system is to find the item for a user. To achieve this purpose, it exploits some models relative to an user, for example: the rates that he gives to other items, the preferences of the user, demographic information associated to an user and context information. Once a good model that relates users and items of the market is built, then we need to find a way with the aim of elaborate this huge amount of information and extract the peculiarities between users and items that permits to achieve a good recommendations. 

