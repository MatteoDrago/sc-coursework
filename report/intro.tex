\section{Theoretical Introduction}

A recommender system is a technology that involves two side of the same coin: \textit{users} and \textit{items}. Items could be products, movies, events, articles that are going to be recommended to users such as: customers, app users, travellers. We can relate the concept of recommendation to a real life situation: if you think to a small shop or boutique, then it's usual that a merchant knew personal preferences of everyday customers; his high quality advices satisfy clients increase profits and visibility of the shop (thank also to the usual word-of-mouth between people).

When dealing with online market places (\href{https://www.netflix.com/}{Netflix}, \href{https://www.amazon.com/}{Amazon}, \href{https://www.ebay.com/}{Ebay} or \href{http://www.asos.com/}{Asos}), personal recommendations, suited up for each particular and different user, can be generated by something that can be seen as an "artificial merchant": the \textbf{recommender system}.

There are several definitions that we can find in the literature and in the \textit{Internet}. Quoting Wikipedia: 

\textit{"A recommender or recommendation system (sometimes replacing "system" with a synonym such as platform or engine) is a subclass of information filtering system that seeks to predict the "rating" or "preference" that a user would give to an item".}

In our opinion such definition is not completely correct; recommender systems are more than what stated. We can think about it as an \textit{personal assistant}, because it can help you in order to make the right decision when you're going to buy something. Moreover, it knows your preferences and tastes; therefore it's not only a machine or an IT system: it perfectly suits you. A more proper definition could be: 

\textit{"A recommender systems is a system that help users discover items they may like, based on a sort of personal knowledge of each user".}

Formally, such systems filter different items after studying a personal profile. They differ from many other information filtering systems, such as standard search engines and \textit{reputation systems}, because recommendations are made on the specific user or group of users. 
We want to stress that the main goal for a recommender system is to show the user what he may like, guiding him on a platform that potentially offers a wide and heterogeneous set of choices. To achieve this purpose, the system can adopt different \textit{user} or \textit{item-based}  models such as: the given rates to different items, preferences on distinct items typologies, either \textit{demographic} or \textit{context} information. 

After the design of a good model, we need to find a way to elaborate this huge amount of information in order to extract the peculiarities between users and items that allows to achieve good recommendations. 

